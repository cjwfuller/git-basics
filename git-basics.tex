\documentclass{beamer}

\mode<presentation> {

\usetheme{default}

%\usecolortheme{orchid}

}

\usepackage{graphicx} % Allows including images
\usepackage{booktabs} % Allows the use of \toprule, \midrule and \bottomrule in tables

%----------------------------------------------------------------------------------------
%	TITLE PAGE
%----------------------------------------------------------------------------------------

\title[Git]{Git - The Basics} % The short title appears at the bottom of every slide, the full title is only on the title page

\author{} % Your name
\institute[] % Your institution as it will appear on the bottom of every slide, may be shorthand to save space
{
 \\ % Your institution for the title page
\medskip
\textit{chris@cjwfuller.com}
}
\date{\today}

\begin{document}

\begin{frame}
\titlepage
\end{frame}

%------------------------------------------------
\section{Introduction}
%------------------------------------------------

\begin{frame}
\frametitle{Git - "the stupid content tracker"}
\textit{``Git is a fast, scalable, distributed revision control system with an unusually rich command set that provides both high-level operations and full access to internals."} - \url{github.com/git/git}
\end{frame}

%------------------------------------------------

\begin{frame}
\frametitle{git status}
Displays paths that have differences between the index file and the current HEAD commit, paths that have differences between the working tree and the index file, and paths in the working tree that are not tracked by Git
\end{frame}

%------------------------------------------------

\begin{frame}[fragile]
\frametitle{git status}
	\scriptsize
	\begin{verbatim}
$ git status
On branch master

Initial commit

Untracked files:
  (use "git add <file>..." to include in what will be committed)

	git-basics.aux
	git-basics.log
	...
	...
	\end{verbatim}
\end{frame}


%------------------------------------------------

\begin{frame}
\frametitle{git add}
Updates the index using the current content found in the working tree, to prepare the content staged for the next commit
\end{frame}

%------------------------------------------------

\begin{frame}[fragile]
\frametitle{git add}
	\scriptsize
	\begin{verbatim}
$ git add git-basics.tex git-basics.pdf
$ git status
On branch master

Initial commit

Changes to be committed:
  (use "git rm --cached <file>..." to unstage)

	new file:   git-basics.pdf
	new file:   git-basics.tex

Untracked files:

	git-basics.aux
	git-basics.log
	...
	...
	\end{verbatim}
\end{frame}

%------------------------------------------------

\begin{frame}
\frametitle{git commit}
Stores the current contents of the index in a new commit along with a log message from the user describing the changes
\end{frame}

%------------------------------------------------

\begin{frame}[fragile]
\frametitle{git commit}
	\scriptsize
	\begin{verbatim}
$ git commit -m "add git-status / git-add slides"
[master (root-commit) d2ab120] add git-status / git-add slides
 2 files changed, 98 insertions(+)
 create mode 100644 git-basics.pdf
 create mode 100644 git-basics.tex
	\end{verbatim}
\end{frame}

%------------------------------------------------

\begin{frame}
\frametitle{git pull}
Incorporates changes from a remote repository into the current branch. In its default mode, git pull is shorthand for git fetch followed by git merge
\end{frame}

%------------------------------------------------

\begin{frame}[fragile]
\frametitle{git pull}
	\scriptsize
	\begin{verbatim}
$ git pull cjwfuller master
remote: Counting objects: 4, done.
remote: Compressing objects: 100% (3/3), done.
remote: Total 4 (delta 0), reused 0 (delta 0), pack-reused 0
Unpacking objects: 100% (4/4), done.
From github.com:cjwfuller/git-basics
 * branch            master     -> FETCH_HEAD
 * [new branch]      master     -> cjwfuller/master
Merge made by the 'recursive' strategy.
 LICENSE   | 22 ++++++++++++++++++++++
 README.md |  1 +
 2 files changed, 23 insertions(+)
 create mode 100644 LICENSE
 create mode 100644 README.md
	\end{verbatim}
\end{frame}

%------------------------------------------------

\begin{frame}
\frametitle{git push}
Updates remote refs using local refs, while sending objects necessary to complete the given refs
\end{frame}

%------------------------------------------------

\begin{frame}[fragile]
\frametitle{git push}
	\scriptsize
	\begin{verbatim}
$ git push cjwfuller master
Counting objects: 7, done.
Delta compression using up to 4 threads.
Compressing objects: 100% (6/6), done.
Writing objects: 100% (6/6), 80.63 KiB | 0 bytes/s, done.
Total 6 (delta 0), reused 0 (delta 0)
To git@github.com:cjwfuller/git-basics.git
   46f9a51..4b14b4b  master -> master
	\end{verbatim}
\end{frame}

%------------------------------------------------

\begin{frame}
\frametitle{git log}
Shows the commit logs
\end{frame}

%------------------------------------------------

\begin{frame}[fragile]
\frametitle{git log}
	\scriptsize
	\begin{verbatim}
$ git log
commit 4b14b4bcebeb6a8532d880b190e0d399edf62e99
Merge: d2ab120 46f9a51
Author: Chris Fuller <chris@cjwfuller.com>
Date:   Sat Mar 7 12:55:00 2015 +0000

    Merge branch 'master' of github.com:cjwfuller/git-basics

commit 46f9a51f1f847c84d5570832208ac269cb22bdce
Author: cjwfuller <chris@cjwfuller.com>
Date:   Sat Mar 7 12:54:14 2015 +0000

    Initial commit

commit d2ab120b641731a9cf288dd3a2449c27c4499300
Author: Chris Fuller <chris@cjwfuller.com>
Date:   Sat Mar 7 12:51:24 2015 +0000

    add git-status / git-add slides
	\end{verbatim}
\end{frame}


%------------------------------------------------

\begin{frame}
\frametitle{git show}
Shows one or more objects (e.g. commits)
\end{frame}

%------------------------------------------------

\begin{frame}[fragile]
\frametitle{git show}
	\tiny
	\begin{verbatim}
$ git show d2ab120b641731a9cf288dd3a2449c27c4499300
commit d2ab120b641731a9cf288dd3a2449c27c4499300
Author: Chris Fuller <chris@cjwfuller.com>
Date:   Sat Mar 7 12:51:24 2015 +0000

    add git-status / git-add slides

diff --git a/git-basics.pdf b/git-basics.pdf
new file mode 100644
index 0000000..e786890
Binary files /dev/null and b/git-basics.pdf differ
diff --git a/git-basics.tex b/git-basics.tex
new file mode 100644
index 0000000..d0102ae
--- /dev/null
+++ b/git-basics.tex
@@ -0,0 +1,98 @@
+\documentclass{beamer}
+
+\mode<presentation> {
+
+\usetheme{default}
+
+%\usecolortheme{orchid}
+
+}
+
+\usepackage{graphicx} % Allows including images
...
...
	\end{verbatim}
\end{frame}


%------------------------------------------------

\begin{frame}
\frametitle{git diff}
Show changes between the things (e.g. two trees)
\end{frame}

%------------------------------------------------

\begin{frame}[fragile]
\frametitle{git diff}
	\tiny
	\begin{verbatim}
$ git diff git-basics.tex
...
...
+On branch master
+Changes to be committed:
+
+       modified:   git-basics.tex
+
+Changes not staged for commit:
+
+       modified:   git-basics.pdf
+
+Untracked files:
...
...
	\end{verbatim}
\end{frame}


%------------------------------------------------

\begin{frame}
\frametitle{git checkout}
Update HEAD to set the specified branch as the current branch
\end{frame}

%------------------------------------------------

\begin{frame}[fragile]
\frametitle{git checkout}
	\tiny
	\begin{verbatim}
$ git status
On branch master
Changes not staged for commit:

	modified:   git-basics.pdf
	modified:   git-basics.tex

Untracked files:

	git-basics.aux
	git-basics.log
	...
	...

$ git checkout git-basics.pdf
$ git status
On branch master
Changes not staged for commit:

	modified:   git-basics.tex

Untracked files:

	git-basics.aux
	git-basics.log
	...
	...

	\end{verbatim}
\end{frame}

%------------------------------------------------

\begin{frame}
\frametitle{git reset}
Reset current HEAD to the specified state
\end{frame}

%------------------------------------------------

\begin{frame}[fragile]
\frametitle{git reset}
	\tiny
	\begin{verbatim}
$ git status
On branch master
Changes not staged for commit:

	modified:   git-basics.pdf
	modified:   git-basics.tex

Untracked files:

	git-basics.aux
	git-basics.log
	...

$ git add git-basics.tex
$ git status
On branch master
Changes to be committed:

	modified:   git-basics.tex

Changes not staged for commit:

	modified:   git-basics.pdf

Untracked files:

	git-basics.aux
	git-basics.log
	...

$ git reset git-basics.tex
Unstaged changes after reset:
M	git-basics.pdf
M	git-basics.tex
	\end{verbatim}
\end{frame}

\end{document} 